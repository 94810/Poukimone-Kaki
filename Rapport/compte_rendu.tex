\documentclass{report}

\usepackage[utf8]{inputenc}
\usepackage[francais]{babel}
\usepackage{graphicx}

\renewcommand{\thesection}{\arabic{section}}
\begin{document}
\title{%
    \begin{minipage}\linewidth
        \centering
        Compte-Rendu TD-03 
        \vskip 3pt
        \large Java-Création d'un simulateur de pokémon
        \author{BOURDELAS Pablo, ROUIBAA Doha, et RYCKAERT Guillaume}
    \end{minipage}
\begin{figure}[ht!]
    \centering
    \includegraphics[width=75mm]{cover.jpg}
\end{figure}
    }   

\maketitle

\section*{Question 1}

Nous avons décidé de découper notre projet en les classes suivantes:
\subsection*{Stats}
Cette classe contient les stats du Poukimone.Chaque Poukimone est est associé a deux instances de cette classe, l'une contenant les Statistiques de base du poukimone (au niveau 1), l'autre contenant celles actuelles (calculées a partir de celles de base).
\subsection*{Poukimone}
Cette classe sert a stocker les caractétistiques permanantes du Poukimone, qui ne changeront jamais au cours du temps.\\ 
Elle contient notemment:
\begin{itemize}
    \item{Le nom du Poukimone}\\
    \item{La courbe d'experience du Poukimone}\\
    \item{Le type du Poukimone}\\
    \item{Le nombre de points d'experience requis pour passer au niveau suivant}\\
    \item{Ses statistiques, stockées via la classe ci-dessus}\\
    \item{Ses Capacités}
\end{itemize}
\subsection*{Ability}
Cette classe sert a gérer les capacités des Poukimones.Elle gère la puissance, la précision,le type, et le nombre de pp de la capcité.
\subsection*{Bag}
Cette classe gère l'inventaire du joueur.Elle gère les potions et les huiles.
\subsection*{Type}
Cette classe gère les types de Poukimone et des Capacités.
\subsection*{Trainer}
Cette classe contient le nom du dresseur, son inventaire, ainsi que son équipe de poukimone.
\subsection*{Fight}
Cette classe sert a gérer le combat.

\begin{figure}[ht!]
    \centering
    \includegraphics[width=75mm]{cover.jpg}
    \caption{"Schéma de classe du projet"}
\end{figure}

\end{document}
